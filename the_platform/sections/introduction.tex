\section{Introduction}
\label{sec:intro}

In 2001, after the the bursting of the dot-com com bubble, the Web 2.0 phenomenon appeared and a survival way to the recent collapse \citep{oreilly2005}.
New applications aimed to involve the user on the content creation process.
One way of engaging the user is to offer a familiar environment, a similar to the one used in a desktop computer.
Web platforms and frameworks made use of \emph{reactive programming} \citep{reactive2014} to achieve a desktop-like experience on the web.
A consequence of turning the content creation responsibility to the user is the massive amount of data created by them, which originated another phenomenon called \emph{big data} \citep{SharmaTWGS14}.
In order to manage all these data, new paradigmas emerged.
\emph{Non-relational} or \emph{No-SQL} \citep{Strauch12} is a database paradigm that manage the data storage and retrieval without using a relational table.
Novel web platforms and frameworks had to incorporate these new aspects in order to model and maintain the complexity requirement of the new applications.

\emph{Meteor.js} \citep{meteor}, or Meteor for short, is an open-source web platform to create applications using the Web 2.0 paradigmas.
The platform provide a reactive approach by focusing on the data flow.
This is done by creating and managing events in the application.
Also, data management is done with a non-SQL database called \emph{MongoDB} \citep{mongo}.
Meteor simplify the application development by providing an unique programming language, javascript, throughout the whole stack process.
Moreover, the platform makes available a set of common tools for business logic and data management.
Finally, Meteor deploys the application in desktop and mobile without needing to change the source code.

In this paper, we seek to give an overview of Meteor aiming to get reader started on the platform.
We start by showing the template system which draws pages and receives event from the user.
Next, the database management, called \emph{collections} in Meteor, is presented and some basic methods for creating, deleting, and updating entries.
Then, we show how to deal with events on the application.
Throughout the paper, we will use one running example which will be a To-do list type.
Lastly, we show how Meteor can be used on resource constrain applications by transferring the heavy workload of the app to a server and letting the user interface respond reactively in a lightweight workload.

The reminder is as follows.
Section \ref{sec:back} shows a quick overview on the history and background informations of the Meteor platform.
Basic concepts and introduction is given in Section \ref{sec:get_star}.
Section \ref{sec:res} shows how Meteor can be used in resource constrain applications.
Conclusions are given in Section \ref{sec:conc}.