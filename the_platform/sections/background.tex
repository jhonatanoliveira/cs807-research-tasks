\section{History and Background}
\label{sec:back}

The Meteor platform was created by a company called Skybreak in 2011 \citep{skybreak}.
Later, in 2012, the company changed its name to Meteor.
The startup was incubated by YCombinator \citep{ycomb} and after receiving an investment of \$11.2 M, the platform development increased considerably.
The platform left beta in October 26th, 2015, with a version that currently provides multiplatform support.

Regarding its internal structure, Meteor is built on top of \emph{Node.js} \citep{node}.
That means that Meteor is driven by events, in order to create an asynchronous model.
This feature is implemented using callback functions: when an event happens a specific function is called to execute a portion of code.
The server-side and client-side ar both implemented using javascript.
In the client-side, templates are used to design user interfaces.
Here, a simple markup language defines the design and the events handles by the application.
Meteor has support for more than one template language, being \emph{Blaze} \citep{meteor} the official one.
In the server-side, Meteor handles data management using collections.
The oficial non-SQL database supported is MongoDB, but new ones are being incorporated in future versions \citep{fathom}.