\documentclass[twoside,11pt]{article}

%%%%% PACKAGES %%%%%%
\usepackage{pgm2016}
\usepackage{amsmath}
\usepackage{algorithm}
\usepackage[noend]{algpseudocode}
\usepackage{subcaption}
\usepackage[utf8]{inputenc}		
\usepackage[english]{babel}		
\usepackage{paralist}			
\usepackage[lowtilde]{url}
\usepackage{fixltx2e}
% Code formatting
\usepackage{listings}
\usepackage{color}
\definecolor{codegreen}{rgb}{0,0.6,0}
\definecolor{codegray}{rgb}{0.5,0.5,0.5}
\definecolor{codepurple}{rgb}{0.58,0,0.82}
\definecolor{backcolour}{rgb}{0.95,0.95,0.92}
\lstdefinestyle{mystyle}{
    backgroundcolor=\color{backcolour},   
    commentstyle=\color{codegreen},
    keywordstyle=\color{magenta},
    numberstyle=\tiny\color{codegray},
    stringstyle=\color{codepurple},
    basicstyle=\footnotesize,
    breakatwhitespace=false,         
    breaklines=true,                 
    captionpos=b,                    
    keepspaces=true,                 
    numbers=left,                    
    numbersep=5pt,                  
    showspaces=false,                
    showstringspaces=false,
    showtabs=false,                  
    tabsize=2
}
\lstset{style=mystyle}

%%%%% MACROS %%%%%%
\algrenewcommand\Return{\State \algorithmicreturn{} }
\algnewcommand{\LineComment}[1]{\State \(\triangleright\) #1}
\renewcommand{\thesubfigure}{\roman{subfigure}}

%%%%% HEADING %%%%%%
% Heading arguments are {volume}{year}{pages}{submitted}{published}{author-full-names}
%\pgmheading{1}{2000}{1-48}{4/00}{10/00}{Cory J. Butz, Jhonatan S. Oliveira, André E. dos Santos, and Anders L. Madsen}


%%%%% SHORT HEADING %%%%%%
% Short headings should be running head and authors last names

%% My commands
\newcommand{\thistitle}{{Inference with Sum-Product Networks}}

\ShortHeadings{\thistitle}{Oliveira}
\firstpageno{1}

\begin{document}

\title{\thistitle}

\author{\name Jhonatan S. Oliveira \email oliveira@cs.uregina.ca \\
\addr Department of Computer Science \\
University of Regina \\ 
Regina, Canada
}

\editor{CS807 - Advanced Architecture - Dr. Gerhard and Trevor Tomesh}

\maketitle

\begin{abstract}%   <- trailing '%' for backward compatibility of .sty file
Sum-product networks (SPNs) are a layer-wise mathematical models for learning and inference with probabilistic graphical model.
The main advantage of SPNs is that a learned model, under certain conditions, is guaranteed to have tractable inference in polynomial time, besides of inference being exact.
Given a SPN, inference is done by propagating up on the network.
In this paper, it is shown how the propagation can be done in parallel.
\end{abstract}

\begin{keywords}
sum-product networks, resource constrain, gpu
\end{keywords}

% Sections
\section{Introduction}
\label{sec:intro}
\section{Critic}
\label{sec:new}

We start by first commenting on the proposed work itself.
SPNs are a very promising PGM.
Its feature to improve learning is of practical value, but the model was mathematically introduced with satisfactory rigour.
The model has internal characteristics which are feasible to test in practice.
These internal characteristics tells when the model will be able or not to run inference in reasonable time.
Thus, one can learn a SPN while watching for these characteristics to be controlled.
In this way, the learned model is guaranteed to do inference.

Besides that, the authors based their work on the arithmetic circuits (ACs) from \cite{Darwiche2009}.
This credit is stated right at the beginning of the paper and later one when comparing their proposed model with related works.
The relation between ACs and SPNs though could be further clarified.
It is possible to detect the similarities from the definitions, but there is a lack of clear motivation that would differentiate an AC model from a SPN.
Indeed, in \cite{Zhao2015} the authors emphasize the relation between ACs and SPNs.
In that work, it is clear that SPNs are defined in a way to be a standalone model.
It is worth mentioning that ACs are formally defined to be a compilation of a given BN.
That is, one could see an AC as an alternative way of telling the BN.
In this sense, defining a SPN as a standalone model implies in having the same expressivity of an AC with the extended feature of modelling any other possible probability distribution that fits into the internal characteristics of a SPN.
This is a very nice point to make in the paper, although it is obfuscated by technical discussion on the model itself.

A personal opinion on the proposed model would be with the lack of semantic on it.
Consider other PGMs, for instance a BN.
One can tell the relationship between variables and even infer other ones by using a BN.
On the contrary, a SPN does not have enough semantic to describe these relations neither a easy way of inferring them.
That is, learning a SPN might be beneficial but the user loses the ability of understanding the learned variables and its interactions.

Now, we comment on the paper itself.
The work is careful described.
This can be seen by the good division and flow of ideas.
The introduction is well motivated and makes an overview of the whole paper.
It also comments on related works, which gives a mor current context to the proposed work.
The Sum-Product Networks section describes the model with formal definitions and common examples.
This makes the new work more easy to follow and show its soundness.
We notice though a small lack of background information on this section.
For instance, the new model is introduced in page 2 of the paper, after a brief paragraph of definitions and contextualization.
The missing background can be of course caused by the lack of space in a conference paper, but this does not affect the work enough to stop the reader of understanding the proposed ideas.
The proofs are short and could have been more smooth, but again this can be due to the lack of space.
Section 3 makes a good comparison with relates models, which emphasize the relevance of the new work and how it overcome the other models' issues.
The following section on Learning Sum-Product networks is just a brief overview of the method, but enough to describe how one can learn a SPN from historical data.
The most impressive section is the experiment results, which shows a SPN beating famous deep architecture models in a practical application of facial completing.
This ends the paper with a very good impression.

The authors also have one section before the conclusion to talk about one possible relation between SPNs and the cortex.
They associate sum and product nodes of a SPN with two different types of cortex neurones.
This serve as a motivation for neuroscience applications that could flourish from the usage of SPNs.
Besides that, it also motivate the study of how this relationship can be beneficial for the SPNs itself.

\section{Conclusions}
\label{sec:conc}

Meteor is a simple platform for building reactive applications on the Web 2.0 paradigma.
Its event drive method offers a real time experience for the users, besides a clear division between client and server sides tasks.
Thus, Meteor can be used in resource constrain products by providing a solution when splitting workload among the more powerful server and the lightweight client.

\vskip 0.2in
\bibliography{references/references}
\end{document}
