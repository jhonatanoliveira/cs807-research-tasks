\section{Introduction}
\label{sec:intro}

\emph{Sum-product networks} (SPNs) \citep{Poon2011} are a new kind of deep architecture to graphically model real world problems.
The paper \citep{Poon2011} won the award for best paper at the 27th Uncertainty in Artificial Intelligence (UAI) conference in 2011.
The UAI is a respected conference in the field of artificial intelligence.
Thus, the paper must be well written and with strong impact in its research area.
Indeed, my critic on this paper is very positive, as reviewed in this paper.
Few negative comments are due to the lack of space, a limitation imposed by the conference, but this does not compromise the high quality of the work.

The key idea in \citep{Poon2011} is to propose a solution for learning and inference in probabilistic graphical models (PGMs).
It is known \citep{koll09} that inference in PGMs are exponential in worse case.
Moreover, learning is also hard to manage given two constrains: the size of the graphical component and the size of the probability information \citep{Zhao2015}.
Thus, SPNs draws inspiration from the work of \cite{Darwiche2009} to solve this issues.
The graphical component is a rooted directed acyclic graph (DAG) and the probability information is saved on some nodes and edges of the DAG.
The DAG has two types of nodes: sum and product, which values are computed by summing or multiplying its children, respectively.
Inference is done by computing all nodes' values going up on this DAG until the root is reached.

My critic for the paper is positive.
The work is overall well described and the presentation is logically well done.
Relating the novel work to related current ones is also well done by the authors, which has a whole section on it.
The only negative critics goes to the brief discussion on background information and the quick passage between sections, without further details on the presented ideas.
But this is acceptable, since it is known that the paper conference has space limitations.

The remainder of this critic follows.
In Section \ref{sec:back} we give a brief introduction to the work presented in the analyzed paper.
The critic is then given in Section \ref{sec:new}.
Section \ref{sec:conc} draws conclusions.