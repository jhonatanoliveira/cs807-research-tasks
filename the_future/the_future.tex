\documentclass[twoside,11pt]{article}

%%%%% PACKAGES %%%%%%
\usepackage{pgm2016}
\usepackage{amsmath}
\usepackage{algorithm}
\usepackage[noend]{algpseudocode}
\usepackage{subcaption}
\usepackage[utf8]{inputenc}		
\usepackage[english]{babel}		
\usepackage{paralist}			
\usepackage[lowtilde]{url}
\usepackage{fixltx2e}
% Code formatting
\usepackage{listings}
\usepackage{color}
\definecolor{codegreen}{rgb}{0,0.6,0}
\definecolor{codegray}{rgb}{0.5,0.5,0.5}
\definecolor{codepurple}{rgb}{0.58,0,0.82}
\definecolor{backcolour}{rgb}{0.95,0.95,0.92}
\lstdefinestyle{mystyle}{
    backgroundcolor=\color{backcolour},   
    commentstyle=\color{codegreen},
    keywordstyle=\color{magenta},
    numberstyle=\tiny\color{codegray},
    stringstyle=\color{codepurple},
    basicstyle=\footnotesize,
    breakatwhitespace=false,         
    breaklines=true,                 
    captionpos=b,                    
    keepspaces=true,                 
    numbers=left,                    
    numbersep=5pt,                  
    showspaces=false,                
    showstringspaces=false,
    showtabs=false,                  
    tabsize=2
}
\lstset{style=mystyle}

%%%%% MACROS %%%%%%
\algrenewcommand\Return{\State \algorithmicreturn{} }
\algnewcommand{\LineComment}[1]{\State \(\triangleright\) #1}
\renewcommand{\thesubfigure}{\roman{subfigure}}

%%%%% HEADING %%%%%%
% Heading arguments are {volume}{year}{pages}{submitted}{published}{author-full-names}
%\pgmheading{1}{2000}{1-48}{4/00}{10/00}{Cory J. Butz, Jhonatan S. Oliveira, André E. dos Santos, and Anders L. Madsen}


%%%%% SHORT HEADING %%%%%%
% Short headings should be running head and authors last names

%% My commands
\newcommand{\thistitle}{{Memcomputing: storing and processing at the same time}}

\ShortHeadings{\thistitle}{Oliveira}
\firstpageno{1}

\begin{document}

\title{\thistitle}

\author{\name Jhonatan S. Oliveira \email oliveira@cs.uregina.ca \\
\addr Department of Computer Science \\
University of Regina \\ 
Regina, Canada
}

\editor{CS807 - Advanced Architecture - Dr. Gerhard and Trevor Tomesh}

\maketitle

\begin{abstract}%   <- trailing '%' for backward compatibility of .sty file
Memcomputing was recently proposed as an alternative to turing paradigma of computation.
The difference here is that the computation and the storage is done within the same architecture.
In order to achieve that, memcopmuters use memelements, which are two-terminal electronics components with resistive, capacitive, and inductive characteristics.
Memcomputers are mathematically proved able to solve NP-hard problems from turing computers.
\end{abstract}

\begin{keywords}
Memcomputing, post-silicon copmuting
\end{keywords}

% Sections
\section{Introduction}
\label{sec:intro}

\emph{Sum-product networks} (SPNs) \citep{Poon2011} are a new kind of deep architecture to graphically model real world problems.
The paper \citep{Poon2011} won the award for best paper at the 27th Uncertainty in Artificial Intelligence (UAI) conference in 2011.
The UAI is a respected conference in the field of artificial intelligence.
Thus, the paper must be well written and with strong impact in its research area.
Indeed, my critic on this paper is very positive, as reviewed in this paper.
Few negative comments are due to the lack of space, a limitation imposed by the conference, but this does not compromise the high quality of the work.

The key idea in \citep{Poon2011} is to propose a solution for learning and inference in probabilistic graphical models (PGMs).
It is known \citep{koll09} that inference in PGMs are exponential in worse case.
Moreover, learning is also hard to manage given two constrains: the size of the graphical component and the size of the probability information \citep{Zhao2015}.
Thus, SPNs draws inspiration from the work of \cite{Darwiche2009} to solve this issues.
The graphical component is a rooted directed acyclic graph (DAG) and the probability information is saved on some nodes and edges of the DAG.
The DAG has two types of nodes: sum and product, which values are computed by summing or multiplying its children, respectively.
Inference is done by computing all nodes' values going up on this DAG until the root is reached.

My critic for the paper is positive.
The work is overall well described and the presentation is logically well done.
Relating the novel work to related current ones is also well done by the authors, which has a whole section on it.
The only negative critics goes to the brief discussion on background information and the quick passage between sections, without further details on the presented ideas.
But this is acceptable, since it is known that the paper conference has space limitations.

The remainder of this critic follows.
In Section \ref{sec:back} we give a brief introduction to the work presented in the analyzed paper.
The critic is then given in Section \ref{sec:new}.
Section \ref{sec:conc} draws conclusions.
\section{Background}
\label{sec:back}

Sum-product networks (SPNs) are based on the ideas of a \emph{network polynomial}.
These are polynomial expression that encodes probability distributions.
They make use of indicator variables attached to each variable of the distribution in order to activate (indicator equals 1) or deactivate (indicator equals 0) each correspondent probability distribution variable.
When all indicators in a network polynomial are set to 1, that is they are all activated, the evaluation of the polynomial yields the unnormalized probability of the evidence.
In other words, if the probability distribution is normalized this evidence value will be 1, otherwise it will be any other real number.
The network polynomial is exponential in the number of variables, but it can be compactly represented in a graph as a SPN.

Formally, a SPN over a set of variables is a rooted directed acyclic graph with leaves being indicator variables and internal nodes being sum or products.
All edges from a sum node has non-negative weights attached to it.
The value of a product node is the product of its children and the value of a sum node is the sum of the product of the value of its children and the weights.
For example, Figure \ref{fig:spn} illustrated a SPN for a naive Bayes model presented in \citep{Poon2011}.
The SPN represents a network polynomial.
Thus, by changing the indicator variable values we set different states of the probability distribution represented by the SPN and, consequently, the network polynomial.
A SPN is then classified as a probabilistic graphical model (PGM), since it is defined by a graph with probability informations encoded.

\begin{figure}[hbt]
    \begin{center}
    \includegraphics[width=0.6\textwidth]{figures/spn.png}
    \caption{A SPN representing a naive Bayes, as shown in \citep{Poon2011}.}
    \label{fig:spn}
    \end{center}
\end{figure}


The key advantages of SPNs are in the learning process, when compared to other PGMs.
Learning a PGM usually involves two steps: first learning the graph structure and later the probability informations.
This method is problematic when trying to optimize the learning process by limiting the size of learning informations \cite{Zhao2015}.
If a restriction is imposed on the graph size, for optimization reasons, it is not guaranteed that the probability informations will also be restricted in size.
The other way around is also true: if a restriction is applied to the size of the learned probability information, it is not guaranteed that the graph will be restricted in size.
This is a practical issue when learning PGMs.
On the other hand, SPNs has the intrinsic feature which guarantees that if the graph is restricted in size the probability information will be as well.
\section{Getting Started}
\label{sec:get_star}

Now, we will follow a common flow for an app creation process.
During this section, we will use a single running example which is an application for todo lists.
This todo example can also be found on the official getting started tutorial in \citep{meteor}.
We will try to keep the same code conventions and names always that possible so the reader can use this paper as an extended guide for that tutorial.
We assume that the Meteor program and required 

The following command line is used to create a new application with Meteor:
\begin{lstlisting}[language=bash]
meteor create todo
\end{lstlisting}
This creates a folder structure as shown below:
\begin{lstlisting}[language=bash]
todo.js todo.html todo.css .meteor
\end{lstlisting}
The \emph{todo.js} file is where the javascript code for the server and client stays.
Here is where the code throws and handles events, besides business logic.
Templates for user interface is done in the \emph{todo.html} file with pure HTML markup language and a template markup language.
For this project Blaze is the template languages used.
The \emph{todo.css} file contains styles for the templates.
Lastly, the \emph{.meteor} stores settings and the meteor aplication itself in a hidden folder.

\subsection{Templates}

Templates are defined using a special tag called \emph{template}.
Once defined, a template can be included within any HTML code.
In this way, a template is a interface module that can be reuses wherever it is required.
All the HTML code and the templates have access to the data made available in that part of the HTML code.
The basic flow is: the javascript code makes available some data to a specific area of the HTML code (or a specific template), so the template language can access that data and print the ones of interest for the user.
Bellow, we show a simple todo list interface.

\begin{lstlisting}[language=html,escapechar=|]
<head>
  <title>Todo List</title>
</head>
 
<body>
  <div class="container">
    <header>
      <h1>Todo List</h1>
      <form class="new-task">  |\label{code:form_b}|
        <input type="text" name="text" placeholder="Type to add new tasks" />
      </form>  |\label{code:form_e}|
    </header>
 
    <ul>
      {{#each tasks}}
        {{> task}} |\label{code:temp}|
      {{/each}}
    </ul>
  </div>
</body>
 
<template name="task">  |\label{code:temp_b}|
  <li class="{{#if checked}}checked{{/if}}"> |\label{code:checked}|
    <button class="delete">&times;</button>  |\label{code:delete}|
    <input type="checkbox" checked="{{checked}}" class="toggle-checked" />
    <span class="text">{{text}}</span>
  </li>
</template>  |\label{code:temp_e}|
\end{lstlisting}

Here, the HTML code defines in its \emph{body} a title and a list.
Notice that Blaze commands are defined within \emph{\{\{} and \emph{\}\}}.
In the list, the Blaze command \emph{\#each} goes through a list of data, as in a loop lace.
The data variable \emph{tasks} contains this list of data and was made available to this part of the code by the javascript code.
The Blaze command \emph{$>$ name\_of\_template} prints a template previously defined with name \emph{name\_of\_template}.
In lines \ref{code:temp_b}-\ref{code:temp_e}, the template named \emph{task} is defined.
This template expect to find available in its scope a data variable called \emph{text}.
In line \ref{code:temp}, the template is included in that spot of the HTML code and the data variable \emph{text} is available on the loop scope of the \emph{\#each} command.
Note that \emph{text} is a field within \emph{tasks}.

On the javascript file we define code for the client and server.
In order to distinguish between them, Meteors makes available a global variable called \emph{isClient}, which goes to true if the running environment is the browser and goes to false if it is the Node.js one.
If a code is called without being under the conditions of a client or server, the code is run in both environments.
Follows the javascript code to create a simple database where the \emph{tasks} are saved.

\begin{lstlisting}[language=java,escapechar=|]
Tasks = new Mongo.Collection("tasks");  |\label{code:db}|
 
if (Meteor.isClient) {  |\label{code:cond}|

  // This code only runs on the client
  Template.body.helpers({  |\label{code:help_b}|
    tasks: function () {  |\label{code:func}|
      return Tasks.find({});
    }  |\label{code:help_e}|
  });
  
  Template.body.events({
    "submit .new-task": function (event) {  |\label{code:event}|
      // Prevent default browser form submit
      event.preventDefault();
 
      // Get value from form element
      var text = event.target.text.value;
 
      // Insert a task into the collection
      Tasks.insert({
        text: text,
        createdAt: new Date() // current time
      });
 
      // Clear form
      event.target.text.value = "";
    }
  });
  
  Template.task.events({
    "click .toggle-checked": function () {  |\label{code:event_checked}|
      // Set the checked property to the opposite of its current value
      Tasks.update(this._id, {
        $set: {checked: ! this.checked}
      });
    },
    "click .delete": function () {  |\label{code:delete_java}|
      Tasks.remove(this._id);
    }
  });
  
}
\end{lstlisting}

Line \ref{code:db} runs on the server and on the client, since it is not inside the conditional in line \ref{code:cond}.
In the server, that command creates a database called \emph{tasks} if it does not exist already.
In the client, it creates a cache of the same database where Meteor manages some saved data in order to reuse repeated queries from the user.
From line \ref{code:help_b} to \ref{code:help_e}, using the global variable \emph{Template}, we make available to the \emph{body} of the HTML whatever data the not named function in line \ref{code:func} return.
The function returns all the entries in the previously created \emph{tasks} database.
These returned data is available to the HTML code on a data variable called \emph{tasks}, as determined in line \ref{code:func}.

\subsection{Inserting Data}

Regarding the data inclusion, this HTML code has a form which will receive the new input data from the user.
Lines \ref{code:form_b}-\ref{code:form_e} defines a simple form where the user can input new tasks.
When pressing enter, an event is thrown from the HTML code and can be handled in the javascript code.

The javascript code watch for an event for the form submission, as shown in line \ref{code:event}.
From here, the default reaction of the browser, which is try to submit the form, is stopped.
The, the value of the input text is saved in a variable.
Next, the cached database variable \emph{Tasks} is used to insert a new entry on the server database with the text saved from the user input.
Finally, the text input is cleared.

Notice in the last insertion process that the cached database was used to update the server database.
This is possible due to the way Meteor works in client and server.
The client has only a cached version of the database.
But whenever this cached version is updated, an event goes to the server (whenever there is connection available) making the server database to update.
The other way around works in the same manner: if the server database is updated, an event goes to all client spreading the update.

\subsection{Updating and Removing Data}

If a task is done, we want to check it out from the list by updating its entry with a done flag.
In the HTML code, line \ref{code:checked}, the template \emph{task} has a conditional statement checking for a variable called \emph{checked}.
If the variable is true, the body of the condition, which is just a string \emph{checked}, is executed.
This variable, as expected, is set in the javascript code.
Indeed, in line \ref{code:event_checked}, the javascript watch for an event of a click on the checkbox.
If it happens, the cached database \emph{Tasks} is updated by setting a field \emph{checked} to the opposite value that the user entered.
Notice the use of a special variable \emph{this} which provides context for HTML access from where the event occurred.


The deletion process is similar to updating but a different function is used on the cached database.
The javascript code, line \ref{code:delete}, watch for the HTML code, line \ref{code:delete}.
When the user clicks in the delete button, the javascript code is executed by removing the correspondent id of the entry.
Again, note the use of \emph{this} as a context variable for the entry where the event occurred.

\section{Resource Constrains}
\label{sec:res}
\section{Conclusions}
\label{sec:conc}

SPNs are layer-wise models used to represent JPDs in a compact and graphical way.
Inference in a SPN can be done by propagating up and down on the SPN DAG.
When propagating in the tree, operations in the same layer can be computed simultaneously, since they do not require input from within the same layer.
In this way, one can perform inference in SPN hopefully faster.
Future works will implement the proposed parallel method to verify improvements.

\vskip 0.2in
\bibliography{references/references}
\end{document}
