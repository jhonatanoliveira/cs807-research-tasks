\section{Background and Related Works}
\label{sec:back}

When defining a new computing paradigma it is necessary to establish crucial fatures, requirements, and implementation comments.
Here we call them criteria of the computing system.
Memcopmuting has 6 criteria as defined in \citep{DiVentra:2012fh}.
Also, \citep{DiVentra:2012fh} says that these criteria are close related to the quantum computing.
Both quantum computing and memcomputing relies in massively parallel computation, but the mechanics behind these requirement are quite different.
Quantum computing relies on a phenomenon called \emph{superposition of states}, while memcomputing relies on the collective dynamic of simple, smaller and classical computers.
