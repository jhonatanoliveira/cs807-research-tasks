\section{Memcomputing Criteria}
\label{sec:new}

This section is drawn from \citep{DiVentra:2012fh} where the criteria were first presented.

The first criteria is a scalable massively parallel architecture with combined information processing and storage.
Here, the memelements are units of processing and storing information, while still working together as a whole system.
A memelement can store information in two forms: its characteristic response, namely the voltage over the element when the input is applied, and in its intrinsic characteristic.
For instance, a capacitor and inductor can store information on its electric and magnetic field, respectively.

Second criteria is the sufficient long information storage times.
The memelements should hold their state, that is the stored information, at least longer than the required time for processing them.
In order to approach that, memelements can make use of non-volatile memory cells.
For instance, the ones using CMOS technology.

The third one is the ability of initialize memory state.
This has to be with the programming desire.
Relevant memelements can be be initialized using a provided mechanism for initialization.
This can be done by applying an input that makes the memelements change their states to extreme points.

The fourth criteria is a mechanism of collective dynamics.
The basic idea here is to promote the collaborative characteristic of the architecture.
That is, the current state of an element depends on the current state of some other elements.
For instance, in a memelement circuit with resistive characteristics changing one element voltage affect other ones, since the resistance changes affecting the voltage in subsequent memelements.

The fifth criteria is the ability to read the final result.
After a processing is done in a memelement architecture, it is desired the possibility of reading the result of it.
But the reading process can not modify the result state itself.
This can be done by choosing an input voltage that the memelements does not change at that level or change little enough to do not be considered.

The sixth and last criteria is the robustness against small imperfections and noise