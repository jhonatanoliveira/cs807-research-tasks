\section{Introduction}
\label{sec:intro}

Commercial computer architectures are based on a turing machine \citep{turing48}.
These computers can generally be split into two main computing tasks: storing and processing.
In order to handle these tasks, different implementation patterns were proposed, for instance, the von-Neumann architecture.
As technology evolved, storage became cheap and processing demand rose to exponential levels.
That is, memories and processors might have different working speed.
Thus, the mode of operation of these architectures imposes a limitations, which are collectively known as the \emph{von-Neumann bottleneck} \citep{Backus:1978:PLV:359576.359579}. 

There are already some suggested alternatives to these limitations, including parallel computing and quantum computing.
Parallel computing are implemented by using multiple core processors.
While parallel computing can minimize the von-Neumann bottleneck issue in one computing unit, when considering scaling the whole workstation it might need specialized units such as \emph{graphic processing units} (GPUs).
These means more complexity on the infra-structure besides considering limitations of the specialized units themselves.
On the other hand, quantum computers are promising in the sense that they are an unique computing platform with intrinsic massively parallel computing scheme.
However, even considering the most recent improvements in the field, a practical quantum computing can not outperform a tradicional one yet.

An alternative computer that might outperform the current ones should be able to has at least two properties: first, an intrinsically massively-parallel architecture, and second, storage and computation are performed by the same basic units.
Transistors can not be used for achieving these goals since they are active elements, that is they require power to perform tasks.
Thus, the massive parallelization comes with the expense of considerable power consumption.
Besides that, normal transistor can not store informations.
Lastly, transistors have low density.

The best electronic element to achieve this goal are memelements.
These are two terminal passive devices, that is do not require energy to work.
Memelements have three different characteristics: resistive, capacitive, and inductive.
These characteristics change accordingly to the current passing through them, as if the characteristic is remembering the current.
Thus the name memelements.
\emph{Memcomputing} \citep{DiVentra:2012fh} is a non-turing paradigma for processing and storing information by using memelements.

The reminder of the paper follows.
Section \ref{sec:back} presents the basic background on the field and relates memcomputing with other alternative to the conventional computer.
The criteria to implement a memcomputing system are given in Section \ref{sec:new}.
Section \ref{sec:app} shows applications already using memcomputing.
Conclusions are drawn in Section \ref{sec:conc}.